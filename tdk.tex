\documentclass[
	%parspace, % Térköz bekezdések közé / Add vertical space between paragraphs
	%noindent, % Bekezdésének első sora ne legyen behúzva / No indentation of first lines in each paragraph
	%nohyp, % Szavak sorvégi elválasztásának tiltása / No hypenation of words
	%twoside, % Kétoldalas nyomtatás / Double sided format
	%final, % Teendők elrejtése / Set final to hide todos
]{elteiktdk}[2020/05/15]

% Dolgozat metaadatai
% Document's metadata
\title{TDK-dolgozat címe}
\date{2020}

% Szerző(k) metaadatai
% Author(s)' metadata
\author{Hallgató Hanga}
\degree{programtervező informatikus MSc}
\period{2. évfolyam}
\coauthor{Hallgató Harold}
\codegree{programtervező informatikus BSc}
\coperiod{3. évfolyam}

% Témavezető(k) metaadatai
% Superivsor(s)' metadata
\supervisor{Témavezető Tamás}
\affiliation{egyetemi tanársegéd}
\cosupervisor{Témavezető Teréz}
\coaffiliation{egyetemi adjunktus}

% Egyetem metaadatai
% University's metadata
\university{Eötvös Loránd Tudományegyetem}
\faculty{Informatikai Kar}
\department{Programozáselmélet és Szoftvertechnológiai Tanszék}
\city{Budapest}
\logo{elte_cimer_szines}

% Irodalomjegyzék hozzáadása
% Add bibliography file
\addbibresource{tdk.bib}

% A dolgozat
% The document
\begin{document}

% Nyelv kiválasztása
% Set document language
\documentlang{magyar}
%\documentlang{english}

% Teendők listája (final dokumentumban nincs)
% List of todos (not in the final document)
\listoftodos[\todolabel]
\cleardoublepage

% Dokumentum beállítások
% Some document settings
\input{settings.tex}

% Fedő- és címlap (kötelező)
% Cover and title page (mandatory)
\makecover
\cleardoublepage
\maketitle

% Tartalomjegyzék (kötelező)
% Table of contents (mandatory)
\tableofcontents
\cleardoublepage

% Tartalom
% Main content
\input{chapters/intro.tex}
\cleardoublepage

\input{chapters/spec.tex}
\cleardoublepage

\input{chapters/impl.tex}
\cleardoublepage

\input{chapters/sum.tex}
\cleardoublepage

% Függelékek (opcionális) - hosszabb részletező táblázatok, sok és/vagy nagy kép esetén hasznos
% Appendices (optional) - useful for detailed information in long tables, many and/or large figures, etc.
\appendix
\input{appendices/sim.tex}
\cleardoublepage

% Irodalomjegyzék (kötelező)
% Bibliography (mandatory)
\addcontentsline{toc}{chapter}{\biblabel}
\printbibliography[title=\biblabel]
\cleardoublepage

% Ábrajegyzék (opcionális) - 3-5 ábra fölött érdemes
% List of figures (optional) - useful over 3-5 figures
\addcontentsline{toc}{chapter}{\lstfigurelabel}
\listoffigures
\cleardoublepage

% Táblázatjegyzék (opcionális) - 3-5 táblázat fölött érdemes
% List of tables (optional) - useful over 3-5 tables
\addcontentsline{toc}{chapter}{\lsttablelabel}
\listoftables
\cleardoublepage

% Forráskódjegyzék (opcionális) - 3-5 kódpélda fölött érdemes
% List of codes (optional) - useful over 3-5 code samples
\addcontentsline{toc}{chapter}{\lstcodelabel}
\lstlistoflistings
\cleardoublepage

% Jelölésjegyzék (opcionális)
% List of symbols (optional)
%\printnomenclature

\end{document}
